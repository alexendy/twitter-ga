%% LyX 2.0.6 created this file.  For more info, see http://www.lyx.org/.
%% Do not edit unless you really know what you are doing.
\documentclass[english]{article}
\usepackage[T1]{fontenc}
\usepackage[utf8]{luainputenc}
\usepackage{babel}
\begin{document}

\title{Twitter Bot}

\maketitle

\section{L'Algo Génétique}


\subsection{La base lexicale initiale}

Plusieurs possibilitées s'offrent à nous


\subsubsection{Initialisée à partir de textes externe}

Des texte de Darwin, des livres érotiques, des discours de Sarkozy,
les grandes heures du stalinisme, des texte conspirationisme ou que
sais-je d'autre... Les hybrides contre natures étant possibles. Ça
permet de partir de textes rigolo et ça s'accorde bien avec un ciblage
de nos lecteur par hashtag


\subsubsection{Initialisée à partir de twitter}

Vraissemblablement sur un hashtag particulier ça nous offre l'avantage
de produire du texte en style twitter mis à part ça ça ne change pas
grand chose


\subsubsection{Pas initialisée}

On peut aussi imaginer qu'initialiser le texte fasse parti du processus,
les bases de données utilisées étant déterminées par l'algo ça offre
des possibilitées de type j'initialise mon générateur en fonction
de ce que dis mon interlocuteur. Paradoxalement si à premiére vu ça
à l'air plus fin j'ai aussi l'impression que ça laisse beaucoup moins
de latitude à l'algo génétique pour optimiser le texte généré.


\subsubsection{Mixer la solution 1.1.1 et la solution 1.1.2}

Ça reste quelque chose d'assez simple


\subsubsection{Mélanger tout}

Bien sur ça rend le machin bien plus compliqué mais l'on peut imaginer
mélanger tout ça


\subsection{Trigger}

Comment l'ago va il décider de produire un twitt ? Il me semble que
la meilleurs solution consiste à utiliser les algos génétiques pour
produire des requétes proches (dans la sytaxe et le fonctionnement
générale'' des requétes SQL. Des trucs du genre donne moi le mot
le plus fréquent dans les requétes utilisant le hastag truc ou au
contraire donne moi le nom de l'utilisateur qui a posté en dernier
sur le hashtag machin. Ça va surement pas être si simple que ça à
construire ce machin.


\subsubsection{Au minimum}

Au minimum deux données seraient extraites de ces requétes SQL:
\begin{itemize}
\item Si il faut ou non produire un twitt
\item Un mot pour initialiser le systéme de génération de texte
\end{itemize}
En écrivant celà je me rend compte que l'abscence de mot peut être
considéré comme l'abscence de twitt.


\subsubsection{Idéalement}

Idéalement il faudrait pouvoir extraire un peu plus d'informations
de ces requétes tel que : le nom d'un ou plusiseurs, un ou plusieurs
hashtags.


\subsection{Mutation et Crossover}


\subsubsection{Mutation}

C'est assez simple à imaginer dans la version la plus simple (avec
juste un graph à faire évoluer) les mutations consiste justent à modifier
certaines valeures de certain liens du graph voir en rajouter ou en
supprimer.

Dans des versions un peu plus avancées ça va consister à changer les
paramétres du trigger


\subsubsection{Crossover}

Là ça me parait bien plus intéressant (et plus compliqué). Dans la
version la plus simple de la chose il va falloir échanger des morceau
de Graph... 

A priori le principe serait d'échager des parties de graph à partir
d'un noeud $N$ et en s'éloignant de ce noeud d'un certain nombre
de niveau $n$ toutes les connexion $N_{c}$ où $N_{c}<n$ de chaque
graph sont échangées entre les deux graphs. pour $N_{c}>n$ les connexions
ne bougent pas. Et pour $N_{c}=n$ il faut déciner ce qu'on fait :p.
On a aussi un probléme : j'ai peur qu'on génére progressivement des
mots complétements isolés : metton qu'Alebert est connecté à C dans
le graph 1 et à en dehors du ``rayon'' de C dans le graph 2. après
l'échange il ne sera plus connecté à rien dans le graph 1 le mot aura
disparu mais l'on n'ajoutera jamais de nouveaux mots...


\subsection{La fitness}


\subsubsection{propres à twitter}

Il me semble que l'on peut utiliser 3 métriques propres à twitter
: les star (favoris), le retweets, et les réponses. Reste à déterminer
comment les répartir


\subsubsection{Apprentissage supervisé}

Rien n'empéche non plus d'évaluer partiellement nous même la qualité
des twitts


\subsection{Structure du bidule}

Partons d'une version minimale qui répond aux twitt d'un hashtag particulier
à partir du moment ou elle trouve un mot contenu 


\subsubsection{Class Chaine Markov / Genotype}


\paragraph{Initialisateur}

Une méthode capable de nous générer un graph sous forme de chaine
de markov à partir d'un hashtag twitter et de nous sauvegarder tout
ça (sous forme de fichier ?)


\paragraph{Loader}

Une méthode pour charger notre chaine de markov à partir d'un fichier
de sauvegarde


\paragraph{Mutation}

Une méthode effectuant des mutations sur notre chaine de markov (changement
des valeurs de certains noeud de la chaine)


\paragraph{Muation nouveau lien}

Création d'un nouveau lien entre deux mots


\paragraph{Mutation supression}

Supression d'un noeud de la chaine


\paragraph{Mutation twitt}

Un systéme de mutation affectant une chaine de Markov à partir d'une
série de twitt


\paragraph{Crossover}

Un méthode effectuant un crossover entre deux graphs (voir plus haut)


\paragraph{Taux de connexion}

Une méthode renvoyant le taux de connexion d'un mot dans le graph
$T=\frac{Connexions_{mot}*NbrMots_{graph}}{Connexions_{graph}}$


\paragraph{Générateur de twitt}

Un générateur de twitt générant ça à partir du graph sans dépasser
une taille de 140 - un certains nombre de caractéres pour la réponse
et éventuellement un tagg pour savoir de quel genotype vient le twitt 


\subsubsection{Algo génétique}


\subparagraph{Trigger}

Un trigger dont le rôle serait de sélectionner alléatoirement un génome
parmi les génomes à évaluer (en commançant par ceux ayant générés
le moins de twitt ?) éventuellement de vérifier si ce génotype est
``compatible'' avec le twitt en question en par exemple vérifiant
qu'il existe au moins un mot avec un taux de connexion suffisant dans
le graph (il faudra préciser ce qu'on entend par suffisant). Puis
si la condition précédente est vérifiée génération d'un twitt (et
sauvegarde quelque part que ce twitt est associé à l'algo machin (dans
le twitt ? ) et sinon répéter l'oppération pour tous les génotypes
n'ayant pas générés suffisament de twitt pour être évalués. 


\paragraph{Calculateur de fitness}

Une méthode pour calculer la fitness d'un genotype. Par exemple $fitness=4*nbr_{r\acute{e}ponses}+2*nbr_{retwitt}+nbr_{favoris}$ 


\paragraph{Gestions des génération}

Un systéme classique de gestion des générations d'individus : évaluation
de la fitness de chaque individus, puis sélection par tournois/élitisme
(à priori rien de spécifique de ce côté)


\subsubsection{Routine}

Une routine pour vérifier réguliérement l'arrivée de nouveaux twitt
(sur le hashtag en question), lancer le trigger pour chacun d'entre
eux et enfin lancer l'évaluation de la fitness de tout ce beau monde
une foi tout les twitt effectués et un laps de temps écoulés (ou un
autre laps de temps plus long écoulé en cas de génotype ne générant
pas suffisament de twitt). 


\subsection{Contraintes spécifiques à twitter }


\subsubsection{Texte limité à 150 caractéres.}


\subsubsection{Choisir notre langue}


\subsection{Problémes potentiels}


\subsubsection{Les autres bots}

Il est bien possible que notre twitt bot se mette rapidement à dialoguer
avec d'autre twittbot : ou pas => houba, ou pas => houba.... le bonheur
!!!! Ce probléme peu nous amener à prendre en compte le nombre d'utilisateurs
différents dans le calcul de la fitness.


\section{La question scientifique }

Pour le moment perso je ne vois pas grand chose d'autre que de comparer
la capacité de générer des IA capables de passer le test de Turing
dans un contexte ou la taille du texte n'est pas limité à un contexte
ou la taille du texte est limité (twitter)... Malheureusement ça me
semble compliqué de réutiliser notre systéme ailleurs que sur twitter
\end{document}
